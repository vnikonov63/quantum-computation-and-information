\documentclass{article}
\usepackage{amsmath, amssymb, framed}  
\usepackage{amsthm}            
\usepackage{graphicx}          
\usepackage{hyperref}          
\usepackage{enumitem}     
\usepackage{braket}
\usepackage[margin=1in]{geometry}  



\title{Quantum Computation and Quantum Information by Michael A. Nielsen and Isaac L. Chuang}
\author{Vasilii Nikonov}
\date{San Diego, CA February 2025}

\begin{document}

\maketitle

\subsection*{Chapter 2: Introduction to quantum mechanics}

\begin{framed} % Creates a visually separated block
    \noindent \textbf{Exercise 2.1: (Linear Dependence Example)}
    
    \medskip
    We can observe that:
    $$
    (1, -1) + (1, 2) - (2, 1) = (0,0) = \boldsymbol{0}
    $$
    Thus, the set of three provided vectors is linearly dependent.
\end{framed}

\bigskip

\begin{framed}
    \noindent \textbf{Exercise 2.2: (Matrix Representations: Example)}

    \medskip
    $V$ is a vector space with basis $\{ \ket{0}, \ket{1} \}$. \\
    $$
        A\ket{0} = \ket{1} \implies A\begin{bmatrix} 1 \\ 0 \end{bmatrix} = \begin{bmatrix} 0 \\ 1 \end{bmatrix} \implies \text{ first column of A is } 
        \ket{1}
    $$
    as multiplication of a 2 by 2 matrix by  $\ket{0}$ is just extracting the first column. Also
    $$
        A\ket{1} = \ket{0} \implies A\begin{bmatrix} 0 \\ 1 \end{bmatrix} = \begin{bmatrix} 1 \\ 0 \end{bmatrix} \implies \text{ second column of A is } \ket{0}
    $$
    Thus $A = \begin{bmatrix} 0 & 1 \\ 1 & 0 \end{bmatrix}$. \\ \\ 
    We know, from (2.7), that $\ket{v_1} = \ket{+} = \begin{bmatrix} \frac{1}{\sqrt{2}} \\ \frac{1}{\sqrt{2}} \end{bmatrix}$ and $\ket{v_2} = \ket{-} = \begin{bmatrix} \frac{1}{\sqrt{2}} \\ \frac{-1}{\sqrt{2}} \end{bmatrix}$ span the $\mathbb{C}^{2}$, they also form a basis for $\mathbb{C}^2$, as they are linearly independent (the only solution for $c_1\ket{+} + c_2\ket{-} = 0$ is trivial). \\ \\ 
    So we can form following linear operators $A_i$ from $\mathbb{C}^{2}$ to $\mathbb{C}^{2}$ :
    \begin{enumerate}
        \item $A_1\ket{0} = \ket{+}$ and $A_1\ket{1} = \ket{-}$
        \item $A_2\ket{0} = \ket{-}$ and $A_2\ket{1} = \ket{+}$
        \item $A_3\ket{+} = \ket{0}$ and $A_3\ket{-} = \ket{1}$
        \item $A_4\ket{+} = \ket{1}$ and $A_4\ket{-} = \ket{0}$
    \end{enumerate}
    We can compute $A_2 = \begin{bmatrix} a & b \\ c & d \end{bmatrix}$, where $ \begin{bmatrix} a & b \\ c & d \end{bmatrix} \begin{bmatrix} 1 \\ 0 \end{bmatrix} = \begin{bmatrix} \frac{1}{\sqrt{2}} \\ \frac{-1}{\sqrt{2}} \end{bmatrix}$ and $ \begin{bmatrix} a & b \\ c & d \end{bmatrix} \begin{bmatrix} 0 \\ 1 \end{bmatrix} = \begin{bmatrix} \frac{1}{\sqrt{2}} \\ \frac{1}{\sqrt{2}} \end{bmatrix}$. We can trivially see, that $A_2 = \begin{bmatrix} \frac{1}{\sqrt{2}} & \frac{1}{\sqrt{2}} \\ \frac{-1}{\sqrt{2}} & \frac{1}{\sqrt{2}} \end{bmatrix}$ \\ \\ 
    A case $A_3 = \begin{bmatrix} e & f \\ g & h \end{bmatrix}$ is a bit more interesting, because we cannot just extract the columns. We can write in matrix form.
    $$
    \begin{bmatrix} e & f \\ g & h \end{bmatrix} \begin{bmatrix} \frac{1}{\sqrt{2}} && \frac{1}{\sqrt{2}} \\ \frac{1}{\sqrt{2}} && \frac{-1}{\sqrt{2}} \end{bmatrix} = I
    $$
    We can multiply both sides on the right by a transpose of the second matrix, as it is orthogonal.
    $$
    A_3 = \begin{bmatrix} e & f \\ g & h \end{bmatrix} = \begin{bmatrix} \frac{1}{\sqrt{2}} && \frac{1}{\sqrt{2}} \\ \frac{1}{\sqrt{2}} && \frac{-1}{\sqrt{2}} \end{bmatrix}
    $$
\end{framed}

\bigskip

\begin{framed}
    \noindent \textbf{Exercise 2.3: (Matrix representation for operator products)}

    \medskip
    We have the following combination of linear operators and vector spaces $V \xrightarrow{A} W \xrightarrow{B} X$. From (2.12) we can write: 
    \begin{equation}
        A\ket{v_i} = \sum_{j}{A_{ji}\ket{w_j}}\text{, and } B\ket{w_j} = \sum_{k}{B_{kj}\ket{x_k}} 
        \tag{1}
    \end{equation}
    $$BA\ket{v_j} = B(\sum_{j}{A_{ji}\ket{v_j}}) \text{ from (1)}$$
    $$
    B(\sum_{j}{A_{ji}\ket{v_j}}) = \sum_{j}{A_{ji}B(\ket{v_j}}) \text{ from linearity of inputs in (2.10)}
    $$
    $$
     = \sum_{j}{A_{ji}\sum_{k}{B_{kj}\ket{x_k}}} \text{ from (1)}
    $$
    \begin{equation}
    = \sum_k(\sum_j{B_{kj}A_{ji}) \ket{x_k}} \text{ rearranging the sum order}
    \tag{2}
    \end{equation}
    If we look from the point of view of a linear operation from V to X, then we need to have some matrix $C$, that would give us:
    $$C\ket{v_i} = \sum_k{C_{ki}\ket{x_k}}$$
    It is precisely 
    $$
    BA_{ki} = \sum_j{B_{kj}A_{ji}} \text{ from (2)}
    $$
\end{framed}

\bigskip

\begin{framed}
    \noindent \textbf{Exercise 2.4: (Matrix representation for identity)}

    \medskip
    
\end{framed}

\bigskip

\begin{framed}
    \noindent \textbf{Exercise 2.5}

    \medskip
\end{framed}

\bigskip

\begin{framed}
    \noindent \textbf{Exercise 2.6}
    
    \medskip
    
    Here we need to show, that any inner product $(\cdot, \cdot)$ is conjugate-linear in the first argument.
    $$
    (\sum_{i}{\lambda_i \ket{w_i}}, \ket{v}) = (\ket{v}, \sum_{i}{\lambda_i \ket{w_i}})^{*} \text{ from (2.13 (2))}
    $$
    $$
    = (\sum_{i}{\lambda_i(\ket{v}, \ket{w_i})})^{*} \text{ from linearity of second inner-product argument (2.13 (1))}
    $$
    $$
    = \sum_{i}{(\lambda_i(\ket{v}, \ket{w_i}))^*} \text{ as conjugate of a sum is a sum of conjugates}
    $$
    $$
    = \sum_{i}{\lambda_i^*(\ket{v}, \ket{w_i})^*}  \text{ as conjugate of a product is a product of conjugates}
    $$
    $$
    = \sum_{i}{\lambda_i^*(\ket{w_i}, \ket{v})} \text{ from (2.13 (2))}
    $$
    
\end{framed}

\bigskip

\begin{framed}
    \noindent \textbf{Exercise 2.7}
    \medskip
    $(\ket{v}, \ket{w}) =  ((1, -1), (1, 1)) = \begin{bmatrix} 1^* -1^* \end{bmatrix} \begin{bmatrix} 1^* \\ 1^* \end{bmatrix} = \begin{bmatrix} 1 -1 \end{bmatrix} \begin{bmatrix} 1 \\ 1 \end{bmatrix} = 1 - 1 = 0 \text{ precisely following (2.14) we establish, that }\ket{w} \text{ and } \ket{v} \text{ are orthogonal}$
    Their normalized forms are $\frac{v}{\| \ket{v \|}} = (\frac{1}{\sqrt{2}}, \frac{-1}{\sqrt{2}})$ and $\frac{w}{\| \ket{w \|}} = (\frac{1}{\sqrt{2}}, \frac{1}{\sqrt{2}})$, as both have a norm of $\sqrt{2}$.
    
\end{framed}

\bigskip

\begin{framed}
    \noindent \textbf{Exercise 2.8}
    
    \medskip
    
    
\end{framed}

\bigskip

\begin{framed}
    \noindent \textbf{Exercise 2.9: (Pauli operators and the outer product)}
    
    \medskip
    
    
\end{framed}

\bigskip

\begin{framed}
    \noindent \textbf{Exercise 2.10}
    
    \medskip
    
    
\end{framed}

\bigskip

\begin{framed}
    \noindent \textbf{Exercise 2.11: (Eigendecomposition of the Pauli matrices)}
    
    \medskip
    \begin{enumerate}
        \item X = $\begin{bmatrix} 0 & 1 \\ 1 & 0 \end{bmatrix} \implies $characteristic equation is $det(\begin{bmatrix} - \lambda & 1 \\ 1 & -\lambda \end{bmatrix}) = \lambda^2 - 1$ and the solution is $\lambda = \pm1$.
        \begin{enumerate}
            \item $\lambda = 1$, Solving $X\ket{v} = \ket{v} \implies \begin{bmatrix} 0 & 1 \\ 1 & 0 \end{bmatrix}\begin{bmatrix} v_1 \\ v_2 \end{bmatrix}=\begin{bmatrix} v_1 \\ v_2 \end{bmatrix} \implies \begin{bmatrix} v_2 \\ v_1 \end{bmatrix}=\begin{bmatrix} v_1 \\ v_2 \end{bmatrix}$. Thus the eigenvector is any scalar multiple of $\begin{bmatrix} 1 \\ 1 \end{bmatrix}$\\
            \item $\lambda = -1$, Solving $X\ket{v} = -\ket{v} \implies \begin{bmatrix} 0 & 1 \\ 1 & 0 \end{bmatrix}\begin{bmatrix} v_1 \\ v_2 \end{bmatrix}=\begin{bmatrix} -v_1 \\ -v_2 \end{bmatrix} \implies \begin{bmatrix} v_2 \\ v_1 \end{bmatrix} = \begin{bmatrix} -v_1 \\ -v_2 \end{bmatrix}$ Thus the eigenvector is any scalar multiple of $\begin{bmatrix} 1 \\ -1 \end{bmatrix}$
        \end{enumerate}
        \item Y = $\begin{bmatrix} 0 & -i \\ i & 0 \end{bmatrix} \implies $characteristic equation is $det(\begin{bmatrix} - \lambda & -i \\ i & -\lambda \end{bmatrix}) = \lambda^2 + 1$ and the solution is $\lambda = \pm i$
        \begin{enumerate}
            \item 
        \end{enumerate}
        \item Z = $\begin{bmatrix} 1 & 0 \\ 0 & -1 \end{bmatrix} \implies $characteristic equation is $det(\begin{bmatrix} - \lambda & 1 \\ 1 & -\lambda \end{bmatrix}) = \lambda^2 - 1$ and the solution is $\lambda = \pm1$
        \begin{enumerate}
            \item 
        \end{enumerate}
    \end{enumerate}
    
    
\end{framed}

\bigskip

\begin{framed}
    \noindent \textbf{Exercise 2.12}
    
    \medskip
    
    
\end{framed}

\bigskip

\begin{framed}
    \noindent \textbf{Exercise 2.13}
    
    \medskip

    $$
    (\ket{w}\bra{v})^{\dagger} = \bra{v}^{\dagger}\ket{w}^{\dagger} \text{ from the fact, that } (AB)^{\dagger} = B^{\dagger}A^{\dagger}. 
    $$
    $$
        \bra{v}^{\dagger}\ket{w}^{\dagger} = \ket{v}\bra{w} \text{ by convention}
    $$
    
\end{framed}

\bigskip

\begin{framed}
    \noindent \textbf{Exercise 2.14: (Anti-linearity of the adjoint)}
    
    \medskip

    $$
    \text{Consider } (\sum_i{a_iA_i})^{*} = \sum_i{(a_iA_i)^{*}} = \sum_i{a_i^*A_i^*} \text{ as conjugation is linear}
    $$

    $$
    \text{ Now consider } ((\sum_i{a_iA_i})^{*})^{T} = (\sum_i{a_i^*A_i^*})^{T} = \sum_i{a_i^*(A_i^*)^{T}} = \sum_i{a_i^*A_i^{\dagger}}\text{ as } a_i \text{ is a scalar }
    $$

    $$
    \text{ Thus we have established, that adjoint operation is anti-linear, namely: } \sum_i{a_iA_i}^{\dagger} = \sum_i{a_i^{*}A_i^{\dagger}}
    $$
    
    
\end{framed}

\bigskip

\begin{framed}
    \noindent \textbf{Exercise 2.15}
    
    \medskip
    
    $(A^{\dagger})^{\dagger} = ((A^{\dagger})^{T})^{*} = (((A^{T})^{*})^{T})^{*} = (((A^{T})^{*})^{*})^{T}$, as taking a conjugate and transposing can easily be interchanged. Conjugating each element in the matrix twice just yields the same initial value, as $(z^{*})^{*} = z , \forall z \in \mathbb{C}$ So we have: $((A^T)^T) = A$ from the definition of the transpose. 
    
\end{framed}

\bigskip

\begin{framed}
    \noindent \textbf{Exercise 2.16}
    
    \medskip
    
    
\end{framed}

\bigskip

\begin{framed}
    \noindent \textbf{Exercise 2.17}
    
    \medskip
    
    
\end{framed}

\bigskip

\begin{framed}
    \noindent \textbf{Exercise 2.18}
    
    \medskip
    
    
\end{framed}

\bigskip

\begin{framed}
    \noindent \textbf{Exercise 2.19: (Pauli matrices: Hermitian and unitary)}
    
    \medskip
    Matrix A is Hermitian if $A^{\dagger} = A$, matrix B is unitary if $B^{\dagger}B = BB^{\dagger} = I$ \\\\
    Now let's consider each of the Pauli matrices 
    \begin{enumerate}
        \item $\sigma_0 = I = \begin{bmatrix} 1 & 0 \\ 0 & 1 \end{bmatrix}$\\\\
        $I^{\dagger} = (I^{T})^{*} = I $ as identity is symmetric, and complex conjugate of a real value is just itself, thus I is Hermitian. \\\\
        $I^{\dagger}I = II = I = II^{\dagger} \implies I$ is unitary

        \item $\sigma_1=\sigma_x=X = \begin{bmatrix} 0 & 1 \\ 1 & 0 \end{bmatrix}$\\\\
        $X^{\dagger} = (X^{T})^* = \begin{bmatrix} 0 & 1 \\ 1 & 0 \end{bmatrix}^{*} = X \implies X$ is Hermitian\\\\
        $X^{\dagger}X = XX = XX^{\dagger} = \begin{bmatrix} 0 & 1 \\ 1 & 0 \end{bmatrix} \begin{bmatrix} 0 & 1 \\ 1 & 0 \end{bmatrix} = \begin{bmatrix} 1 & 0 \\ 0 & 1 \end{bmatrix} = I \implies X$ is unitary

        

        \item $\sigma_2=\sigma_y=Y = \begin{bmatrix} 0 & -i \\ i & 0 \end{bmatrix}$\\\\
        $Y^{\dagger} = (Y^{T})^* = \begin{bmatrix} 0 & i \\ -i & 0 \end{bmatrix}^{*} = \begin{bmatrix} 0 & -i \\ i & 0 \end{bmatrix} = Y \implies Y$ is Hermitian\\\\
        $Y^{\dagger}Y = YY = YY^{\dagger} = \begin{bmatrix} 0 & -i \\ i & 0 \end{bmatrix}\begin{bmatrix} 0 & -i \\ i & 0 \end{bmatrix}=\begin{bmatrix} -i^2 & 0 \\ 0 & -i^2 \end{bmatrix}=\begin{bmatrix} 1 & 0 \\ 0 & 1 \end{bmatrix} = I \implies Y$ is unitary

        \item $\sigma_3=\sigma_z=Z = \begin{bmatrix} 1 & 0 \\ 0 & -1 \end{bmatrix}$\\\\
        $Z^{\dagger} = (Z^{T})^{*} = \begin{bmatrix} 1 & 0 \\ 0 & -1 \end{bmatrix}^{*} = Z \implies Z$ is Hermitian\\\\
        $Z^{\dagger}Z = ZZ = ZZ^{\dagger} = \begin{bmatrix} 1 & 0 \\ 0 & -1 \end{bmatrix}\begin{bmatrix} 1 & 0 \\ 0 & -1 \end{bmatrix}=\begin{bmatrix} 1 & 0 \\ 0 & (-1)^2 \end{bmatrix}=I \implies Z$ is unitary
    \end{enumerate}
    
\end{framed}

\bigskip

\begin{framed}
    \noindent \textbf{Exercise 2.20: (Basis changes)}
    
    \medskip
    
    
\end{framed}

\bigskip

\begin{framed}
    \noindent \textbf{Exercise 2.21}
    
    \medskip
    
    
\end{framed}

\bigskip

\begin{framed}
    \noindent \textbf{Exercise 2.22}
    
    \medskip
    
    
\end{framed}

\bigskip

\begin{framed}
    \noindent \textbf{Exercise 2.23}
    
    \medskip
    
    
\end{framed}

\bigskip

\begin{framed}
    \noindent \textbf{Exercise 2.24: (Hermiticity of positive operators)}
    
    \medskip
    
    
\end{framed}

\bigskip

\begin{framed}
    \noindent \textbf{Exercise 2.25}
    
    \medskip
    
    
\end{framed}

\bigskip

\begin{framed}
    \noindent \textbf{Exercise 2.26}
    $$\ket{\psi}^{\otimes2} = \ket{\psi} \otimes \ket{\psi} \text{, analogously } \ket{\psi}^{\otimes3} = \ket{\psi} \otimes \ket{\psi} \otimes \ket{\psi}$$
    Explicitly: 
    $$
    \ket{\psi}^{\otimes2} = (\frac{1}{\sqrt{2}}(\ket{0} + \ket{1})) \otimes (\frac{1}{\sqrt{2}}(\ket{0} + \ket{1})) = \frac{1}{2}((\ket{0} + \ket{1}) \otimes (\ket{0} + \ket{1})) \text{ from (2.42) }
    $$
    $$
    = \frac{1}{2}(\ket{0} \otimes \ket{0} + \ket{1} \otimes \ket{0} + \ket{0} \otimes \ket{1} + \ket{1} \otimes \ket{1}) \text{ from distributive properties (2.43. 2.44)}
    $$
    Using Kronecker product:
    $$
     \ket{0} = \begin{bmatrix}1 \\ 0\end{bmatrix}, \ket{1} = \begin{bmatrix}0 \\ 1\end{bmatrix}, so \ket{\psi}^{\otimes2} = \begin{bmatrix} \frac{1}{\sqrt{2}} \\ \frac{1}{\sqrt{2}}\end{bmatrix} \otimes \begin{bmatrix}\frac{1}{\sqrt{2}} \\ \frac{1}{\sqrt{2}}\end{bmatrix} = \begin{bmatrix}\frac{1}{\sqrt{2}}\begin{bmatrix}\frac{1}{\sqrt{2}} \\ \frac{1}{\sqrt{2}}\end{bmatrix} \\ \frac{1}{\sqrt{2}}\begin{bmatrix}\frac{1}{\sqrt{2}} \\ \frac{1}{\sqrt{2}}\end{bmatrix}\end{bmatrix} = \frac{1}{2}\begin{bmatrix}1 \\ 1\\ 1\\ 1\end{bmatrix}
    $$
    Explicitly: 
    $$
    \ket{\psi}^{\otimes3} = (\frac{1}{\sqrt{2}}(\ket{0} + \ket{1})) \otimes (\frac{1}{\sqrt{2}}(\ket{0} + \ket{1})) \otimes (\frac{1}{\sqrt{2}}(\ket{0} + \ket{1})) = \frac{1}{2\sqrt{2}}((\ket{0} + \ket{1}) \otimes (\ket{0} + \ket{1})) \otimes (\ket{0} + \ket{1}))
    $$
    $$
    = \frac{1}{2\sqrt{2}}(\ket{0} \otimes \ket{0} \otimes \ket{0} + \ket{0} \otimes \ket{0} \otimes \ket{1} + \ket{0} \otimes \ket{1} \otimes \ket{0} + \ket{0} \otimes \ket{1} \otimes \ket{1} + \ket{1} \otimes \ket{0} \otimes \ket{0} + \ket{1} \otimes \ket{0} \otimes \ket{1} + 
    $$
    $$
    + \ket{1} \otimes \ket{1} \otimes \ket{0} + \ket{1} \otimes \ket{1} \otimes \ket{1})
    $$
    Using Kronecker product:
    $$
    \ket{\psi}^{\otimes3} = \ket{\psi}^{\otimes2} \otimes \ket{\psi} = (\frac{1}{2}\begin{bmatrix}1 \\ 1\\ 1\\ 1\end{bmatrix}) \otimes \begin{bmatrix} \frac{1}{\sqrt{2}}\\\frac{1}{\sqrt{2}}\end{bmatrix} = \frac{1}{2\sqrt{2}}\begin{bmatrix}1 \\ 1\\ 1\\ 1\end{bmatrix} \otimes \begin{bmatrix}1 \\1 \end{bmatrix} = \frac{1}{2\sqrt{2}}\begin{bmatrix}1 \\ 1\\ 1\\ 1\\ 1\\ 1\\ 1\\ 1\\ 1\end{bmatrix}
    $$
    \medskip
    
    
\end{framed}

\bigskip

\begin{framed}
    \noindent \textbf{Exercise 2.27}
    
    \medskip
    
    
\end{framed}

\bigskip

\begin{framed}
    \noindent \textbf{Exercise 2.28}
    
    \medskip
    
    
\end{framed}

\bigskip

\begin{framed}
    \noindent \textbf{Exercise 2.29}
    
    \medskip
    
    
\end{framed}

\bigskip

\begin{framed}
    \noindent \textbf{Exercise 2.30}
    
    \medskip
    
    
\end{framed}

\bigskip

\begin{framed}
    \noindent \textbf{Exercise 2.31}
    
    \medskip
    
    
\end{framed}

\bigskip

\begin{framed}
    \noindent \textbf{Exercise 2.32}
    
    \medskip
    
    
\end{framed}

\bigskip

\begin{framed}
    \noindent \textbf{Exercise 2.33}
    
    \medskip
    
    
\end{framed}

\bigskip

\begin{framed}
    \noindent \textbf{Exercise 2.34}
    
    \medskip
    
    
\end{framed}

\bigskip

\begin{framed}
    \noindent \textbf{Exercise 2.35: (Exponential of the Pauli matrices)}
    
    \medskip
    
    
\end{framed}

\bigskip

\begin{framed}
    \noindent \textbf{Exercise 2.36}
    
    \medskip
    
    
\end{framed}

\bigskip

\begin{framed}
    \noindent \textbf{Exercise 2.37: (Cyclic property of the trace)}
    
    \medskip
    
    
\end{framed}

\bigskip

\begin{framed}
    \noindent \textbf{Exercise 2.38: (Linearity of the trace)}
    
    \medskip
    
    
\end{framed}

\bigskip

\begin{framed}
    \noindent \textbf{Exercise 2.39: (The Hilbert–Schmidt inner product on operators)}
    
    \medskip
    
    
\end{framed}

\bigskip

\begin{framed}
    \noindent \textbf{Exercise 2.40: (Commutation relations for the Pauli matrices)}
    
    \medskip
    \begin{enumerate}
        \item $[X, Y] = XY - YX = \begin{bmatrix}0 & 1 \\ 1 & 0 \end{bmatrix}\begin{bmatrix}0 & -i \\ i & 0\end{bmatrix}-\begin{bmatrix}0 & -i \\ i & 0\end{bmatrix}\begin{bmatrix}0 & 1 \\ 1 & 0 \end{bmatrix} = \begin{bmatrix}i & 0 \\ 0 & -i \end{bmatrix}-\begin{bmatrix}-i & 0 \\ 0 & i \end{bmatrix}= \begin{bmatrix}2i & 0 \\ 0 & -2i \end{bmatrix} = 2i\begin{bmatrix}1 & 0 \\ 0 & -1\end{bmatrix} = 2iZ$
        \item $[Y, Z] = YZ - ZY = \begin{bmatrix}0 & -i \\ i & 0 \end{bmatrix}\begin{bmatrix}1 & 0 \\ 0 & -1\end{bmatrix} - \begin{bmatrix}1 & 0 \\ 0 & -1\end{bmatrix}\begin{bmatrix}0 & -i \\ i & 0 \end{bmatrix} = \begin{bmatrix}0 & i \\ i & 0\end{bmatrix} - \begin{bmatrix}0 & -i \\ -i & 0\end{bmatrix} = \begin{bmatrix}0 & 2i \\ 2i & 0\end{bmatrix} = 2i\begin{bmatrix}0 & 1 \\ 1 & 0\end{bmatrix} = 2iX$
        \item $[Z, X] = ZX - XZ = \begin{bmatrix}1 & 0 \\ 0 & -1\end{bmatrix}\begin{bmatrix}0 & 1 \\ 1 & 0 \end{bmatrix} - \begin{bmatrix}0 & 1 \\ 1 & 0 \end{bmatrix}\begin{bmatrix}1 & 0 \\ 0 & -1\end{bmatrix} = \begin{bmatrix}0 & 1 \\ -1 & 0\end{bmatrix} - \begin{bmatrix}0 & -1 \\ 1 & 0\end{bmatrix} = \begin{bmatrix}0 & 2 \\ -2 & 0\end{bmatrix} = 2i\begin{bmatrix}0 & -i \\ i & 0\end{bmatrix} = 2iY$
    \end{enumerate}
    Note, that $\epsilon_{jkl}$ in (2.74) is Levi-Civita symbol.
\end{framed}

\bigskip

\begin{framed}
    \noindent \textbf{Exercise 2.41: (Anti-commutation relations for the Pauli matrices)}
    
    \medskip
    \begin{enumerate}
        \item $\{\sigma_1, \sigma_2\} = XY + YX = \begin{bmatrix}0 & 1 \\ 1 & 0\end{bmatrix} \begin{bmatrix}0 & -i \\ i & 0\end{bmatrix} + \begin{bmatrix}0 & -i \\ i & 0\end{bmatrix} \begin{bmatrix}0 & 1 \\ 1 & 0\end{bmatrix} = \begin{bmatrix}i & 0 \\ 0 & -i\end{bmatrix} + \begin{bmatrix}-i & 0 \\ 0 & i\end{bmatrix} = \begin{bmatrix}0 & 0 \\ 0 & 0\end{bmatrix} = 0$
        \item $\{\sigma_2, \sigma_3\} = YZ + ZY = \begin{bmatrix}0 & -i \\ i & 0\end{bmatrix} \begin{bmatrix}1 & 0 \\ 0 & -1\end{bmatrix} + \begin{bmatrix}1 & 0 \\ 0 & -1\end{bmatrix} \begin{bmatrix}0 & -i \\ i & 0\end{bmatrix} = \begin{bmatrix}0 & -i \\ -i & 0\end{bmatrix} + \begin{bmatrix}0 & i \\ i & 0\end{bmatrix} = \begin{bmatrix}0 & 0 \\ 0 & 0\end{bmatrix} = 0$
        \item $\{\sigma_3, \sigma_1\} = ZX + XZ = \begin{bmatrix}1 & 0 \\ 0 & -1\end{bmatrix} \begin{bmatrix}0 & 1 \\ 1 & 0\end{bmatrix} + \begin{bmatrix}0 & 1 \\ 1 & 0\end{bmatrix} \begin{bmatrix}1 & 0 \\ 0 & -1\end{bmatrix} = \begin{bmatrix}0 & 1 \\ -1 & 0\end{bmatrix} + \begin{bmatrix}0 & -1 \\ 1 & 0\end{bmatrix} = \begin{bmatrix}0 & 0 \\ 0 & 0\end{bmatrix} = 0$
        \item $\sigma_1^2  = XX= \begin{bmatrix}0 & 1 \\ 1 & 0\end{bmatrix} \begin{bmatrix}0 & 1 \\ 1 & 0\end{bmatrix} = \begin{bmatrix}1 & 0 \\ 0 & 1\end{bmatrix} = I$
        \item $\sigma_2^2  = YY= \begin{bmatrix}0 & -i \\ i & 0\end{bmatrix} \begin{bmatrix}0 & -i \\ i & 0\end{bmatrix} = \begin{bmatrix}1 & 0 \\ 0 & 1\end{bmatrix} = I$
        \item $\sigma_3^2  = ZZ= \begin{bmatrix}1 & 0 \\ 0 & -1\end{bmatrix} \begin{bmatrix}1 & 0 \\ 0 & -1\end{bmatrix} = \begin{bmatrix}1 & 0 \\ 0 & 1\end{bmatrix} = I$


    \end{enumerate}
    
\end{framed}

\bigskip

\begin{framed}
    \noindent \textbf{Exercise 2.42}
    
    \medskip
    $$
    \text{Consider }  \frac{[A,B] +\{A,B\}}{2} = \frac{AB - BA + AB + BA}{2} \text{ from (2.66), (2.67)}
    $$
    $$
    = \frac{2AB}{2} = AB
    $$
    
\end{framed}

\bigskip

\begin{framed}
    \noindent \textbf{Exercise 2.43}
    
    \medskip
    \begin{equation*}
        \text{Note, that Kronicker delta is defined as } \delta_{jk} =
        \begin{cases}
            0, & \text{if } j \neq k, \\
            1, & \text{if } j = k.
        \end{cases}
    \end{equation*}
    \begin{equation*}
        \text{1. Consider the case } j\neq k \text{, then from (2.74, 2.75) } [\sigma_j, \sigma_k] + \{\sigma_j, \sigma_k \} = 2i\sum_{l=1}^{3}\epsilon_{jkl}\sigma_l + 0
    \end{equation*}
    \begin{equation*}
        \text{Expanding the commutator and anti-commutator using their definitions we get: }
    \end{equation*}
    \begin{equation*}
        \sigma_j\sigma_k - \sigma_k\sigma_j + \sigma_j\sigma_k + \sigma_k\sigma_j = 2i\sum_{l=1}^{3}\epsilon_{jkl}\sigma_l \implies 2\sigma_j\sigma_k = 2i\sum_{l=1}^{3}\epsilon_{jkl}\sigma_l \implies
        \sigma_j\sigma_k = i\sum_{l=1}^{3}\epsilon_{jkl}\sigma_l
    \end{equation*}
    \begin{equation*}
        \delta_{jk} = 0 \text{ as } j \neq k \implies \delta_{jk}I = 0 \implies \sigma_j\sigma_k = \delta_{jk}I + i\sum_{l=1}^{3}\epsilon_{jkl}\sigma_l
    \end{equation*}
    \begin{equation*}
        \text{2. Now consider the case } j = k \implies \delta_{jk} = 1 \text{ and from (2.76) } \sigma_{jk}\sigma_{jk} = I \implies \sigma_{jk} = \delta_{jk}I.
    \end{equation*}
    \begin{equation*}
        \text{Consider }\sigma_j\sigma_k = 2i\sum_{l=1}^{3}\epsilon_{jkl}\sigma_l = 2i(\epsilon_{jk1}\sigma_1+\epsilon_{jk2}\sigma_2+\epsilon_{jk2}\sigma_3z) = 2i(0) = 0
    \end{equation*}
    \begin{equation*}
        \text{ because in all the cases } \epsilon \text{ would have a repeating index, as } j = k.
    \end{equation*}
    \begin{equation*}
        \text{ We have established, that } \sigma_{jk} = \delta_{jk}I + \sum_{l=1}^3{\epsilon_{jkl}\sigma_l} \text{ } \forall j,k
    \end{equation*}

    
\end{framed}

\bigskip

\begin{framed}
    \noindent \textbf{Exercise 2.44}
    $$
    [A,B] = 0 \implies AB - BA = 0, \{A, B\} = 0 \implies AB + BA = 0 \text{ from definitions (2.66, 2.67)}
    $$
    $$
    \text{Then their sum } [A,B] + \{A,B\} = 2AB = 0 \text{ We now multiply both sides on the left by } A^{-1} \text{}
    $$
    $$
    A^{-1}2AB = A^{-1}0 \implies 2B = 0 \implies B = 0.
    $$
    \medskip
    
    
\end{framed}


\bigskip

\begin{framed}
    \noindent \textbf{Exercise 2.45}
    
    \medskip
    
    Consider Hermitian conjugate of a commutator between two operators A and B 
    $$
    [A,B]^{\dagger} = (AB - BA)^{\dagger} \text{ from (2.66)}
    $$
    $$
    = ((AB - BA)^{T})^{*} \text{ from definition of Hermitian conjugate}
    $$
    $$
    = ((AB)^{T} - (BA)^{T}) ^ {*} = (B^TA^T - A^TB^T)^* = (B^T)^*(A^T)^* - (A^T)^*(B^T)^*
    $$
    $$
    = B^{\dagger}A^{\dagger} - A^{\dagger}B^{\dagger} = [B^{\dagger},A^{\dagger}]
    $$
\end{framed}

\bigskip
\begin{framed}
\noindent \textbf{Exercise 2.46}

\medskip

Consider $[A, B] = AB - BA = -(-AB + BA) = - (BA - AB) = -[B, A]$
\end{framed}

\bigskip

\begin{framed}
    \noindent \textbf{Exercise 2.47}
    
    \medskip

    $$
    A, B \text{ are Hermitian, then each is equal to their conjugate transpose } A = A^{\dagger}, B = B^{\dagger}
    $$
    $$
    \text{Consider } (i[A, B])^{\dagger} = i^{*}[A,B]^{\dagger} \text{ as } (cA)^{\dagger} = c^{*}A^{\dagger} \text{ where } c \text{ is a complex scalar}.
    $$
    $$
    = (-i)[A, B]^{\dagger} = (-i)[B^{\dagger},A^{\dagger}] \text{ from Exercise 2.45}
    $$
    $$
    = (-i)[B, A] = -(i)(-[A,B]) \text{ from exercise 2.46} = i[A,B] \implies i[A,B] \text{ is Hermitian}
    $$
\end{framed}

\bigskip

\begin{framed}
    \noindent \textbf{Exercise 2.48}
    
    \medskip
    
    
\end{framed}

\bigskip

\begin{framed}
    \noindent \textbf{Exercise 2.49}
    
    \medskip
    
    
\end{framed}

\bigskip

\begin{framed}
    \noindent \textbf{Exercise 2.50}
    
    \medskip
    
    
\end{framed}

\bigskip

\begin{framed}
    \noindent \textbf{Exercise 2.51}
    
    \medskip
    $$
    \text{Consider } HH^{\dagger} = \frac{1}{\sqrt{2}}\begin{bmatrix}1 & 1\\ 1 & -1\end{bmatrix}\frac{1}{\sqrt{2}}\begin{bmatrix}1 & 1\\ 1 & -1\end{bmatrix}=\frac{1}{2}\begin{bmatrix}2 & 0 \\ 0 & 2\end{bmatrix}=I
    $$
    $$
    \text{Consider }H^{\dagger}H = \frac{1}{\sqrt{2}}\begin{bmatrix}1 & 1\\ 1 & -1\end{bmatrix}\frac{1}{\sqrt{2}}\begin{bmatrix}1 & 1\\ 1 & -1\end{bmatrix}=\frac{1}{2}\begin{bmatrix}2 & 0 \\ 0 & 2\end{bmatrix}=I
    $$
    Thus H is indeed unitary.
\end{framed}

\bigskip

\begin{framed}
    \noindent \textbf{Exercise 2.52}
    
    \medskip

    $$
    H^2 = \frac{1}{2}\begin{bmatrix}1 & 1\\ 1 & -1\end{bmatrix}\begin{bmatrix}1 & 1\\ 1 & -1\end{bmatrix} = \frac{1}{2}\begin{bmatrix} 1*1 + 1*1 & 1*1 - 1*1 \\ 1 * 1 - 1 * 1 & 1 * 1 + (-1)*(-1)\end{bmatrix} = \frac{1}{2}\begin{bmatrix}2 & 0 \\ 0 & 2\end{bmatrix} = I
    $$
    
\end{framed}

\bigskip

\begin{framed}
    \noindent \textbf{Exercise 2.53}
    
    \medskip
    
    
\end{framed}

\bigskip

\begin{framed}
    \noindent \textbf{Exercise 2.54}
    
    \medskip
    
    
\end{framed}

\bigskip

\begin{framed}
    \noindent \textbf{Exercise 2.55}
    
    \medskip
    
    
\end{framed}

\bigskip

\begin{framed}
    \noindent \textbf{Exercise 2.56}
    
    \medskip
    
    
\end{framed}

\bigskip

\begin{framed}
    \noindent \textbf{Exercise 2.57: (Cascaded measurements are single measurements)}
    
    \medskip
    
    
\end{framed}

\bigskip

\begin{framed}
    \noindent \textbf{Exercise 2.58}
    
    \medskip
    
    
\end{framed}

\bigskip

\begin{framed}
    \noindent \textbf{Exercise 2.59}
    
    \medskip
    
    
\end{framed}

\bigskip

\begin{framed}
    \noindent \textbf{Exercise 2.60}
    
    \medskip
    
    
\end{framed}

\bigskip

\begin{framed}
    \noindent \textbf{Exercise 2.61}
    
    \medskip
    
    
\end{framed}

\bigskip

\begin{framed}
    \noindent \textbf{Exercise 2.62}
    
    \medskip
    
    
\end{framed}

\bigskip

\begin{framed}
    \noindent \textbf{Exercise 2.63}
    
    \medskip
    
    
\end{framed}

\bigskip

\begin{framed}
    \noindent \textbf{Exercise 2.64}
    
    \medskip
    
    
\end{framed}

\bigskip

\begin{framed}
    \noindent \textbf{Exercise 2.65}
    
    \medskip
    
    
\end{framed}

\bigskip

\begin{framed}
    \noindent \textbf{Exercise 2.66}
    
    \medskip
    
    
\end{framed}

\bigskip

\begin{framed}
    \noindent \textbf{Exercise 2.67}
    
    \medskip
    
    
\end{framed}

\bigskip

\begin{framed}
    \noindent \textbf{Exercise 2.68}
    
    \medskip
    
    
\end{framed}

\bigskip

\begin{framed}
    \noindent \textbf{Exercise 2.69}
    
    \medskip
    
    
\end{framed}

\bigskip

\begin{framed}
    \noindent \textbf{Exercise 2.70}
    
    \medskip
    
    
\end{framed}

\bigskip

\begin{framed}
    \noindent \textbf{Exercise 2.71: (Criterion to decide if a state is mixed or pure)}
    
    \medskip
    
    
\end{framed}

\bigskip

\begin{framed}
    \noindent \textbf{Exercise 2.72: (Bloch sphere for mixed states)}
    
    \medskip
    
    
\end{framed}

\bigskip

\begin{framed}
    \noindent \textbf{Exercise 2.73}
    
    \medskip
    
    
\end{framed}

\bigskip

\begin{framed}
    \noindent \textbf{Exercise 2.74}
    
    \medskip
    
    
\end{framed}

\bigskip

\begin{framed}
    \noindent \textbf{Exercise 2.75}
    
    \medskip
    
    
\end{framed}

\bigskip

\begin{framed}
    \noindent \textbf{Exercise 2.76}
    
    \medskip
    
    
\end{framed}

\bigskip

\begin{framed}
    \noindent \textbf{Exercise 2.77}
    
    \medskip
    
    
\end{framed}

\bigskip

\begin{framed}
    \noindent \textbf{Exercise 2.78}
    
    \medskip
    
    
\end{framed}

\bigskip

\begin{framed}
    \noindent \textbf{Exercise 2.79}
    
    \medskip
    
    
\end{framed}

\bigskip

\begin{framed}
    \noindent \textbf{Exercise 2.80}
    
    \medskip
    
    
\end{framed}

\bigskip

\begin{framed}
    \noindent \textbf{Exercise 2.81: (Freedom in purifications)}
    
    \medskip
    
    
\end{framed}

\bigskip

\begin{framed}
    \noindent \textbf{Exercise 2.82}
    
    \medskip
    
    
\end{framed}

\bigskip

\begin{framed}
    \noindent \textbf{End of Chapter Exercise 2.1: (Functions of the Pauli matrices)}
    
    \medskip
    
    
\end{framed}

\bigskip

\begin{framed}
    \noindent \textbf{End of Chapter Exercise 2.2: (Properties of the Schmidt number)}
    
    \medskip
    
    
\end{framed}

\bigskip

\begin{framed}
    \noindent \textbf{End of Chapter Exercise 2.3: (Tsirelson’s inequality)}
    
    \medskip
    
    
\end{framed}

\subsection*{Chapter 3: Introduction to computer science}

\bigskip

\begin{framed}
    \noindent \textbf{Exercise 3.1: Non-computable processes in Nature}
    
    \medskip
    
    
\end{framed}

\bigskip

\begin{framed}
    \noindent \textbf{Exercise 3.2: Turing numbers}
    
    \medskip
    
    
\end{framed}

\bigskip

\begin{framed}
    \noindent \textbf{Exercise 3.3: Turing machine to reverse a bit string}
    
    \medskip
    
    
\end{framed}

\bigskip

\begin{framed}
    \noindent \textbf{Exercise 3.4: Turing machine to add modulo 2}
    
    \medskip
    
    
\end{framed}

\bigskip

\begin{framed}
    \noindent \textbf{Exercise 3.5: Halting problem with no inputs}
    
    \medskip
    
    
\end{framed}

\bigskip

\begin{framed}
    \noindent \textbf{Exercise 3.6: Probabilistic halting problem}
    
    \medskip
    
    
\end{framed}

\bigskip

\begin{framed}
    \noindent \textbf{Exercise 3.7: Halting oracle}
    
    \medskip
    
    
\end{framed}

\bigskip

\begin{framed}
    \noindent \textbf{Exercise 3.8: Universality of NAND}
    
    \medskip
    
    
\end{framed}

\bigskip

\begin{framed}
    \noindent \textbf{Exercise 3.9: }
    
    \medskip
    
    
\end{framed}

\bigskip

\begin{framed}
    \noindent \textbf{Exercise 3.10: }
    
    \medskip
    
    
\end{framed}

\bigskip

\begin{framed}
    \noindent \textbf{Exercise 3.11: }
    
    \medskip
    
    
\end{framed}

\bigskip

\begin{framed}
    \noindent \textbf{Exercise 3.12: }
    
    \medskip
    
    
\end{framed}

\bigskip

\begin{framed}
    \noindent \textbf{Exercise 3.13: }
    
    \medskip
    
    
\end{framed}

\bigskip

\begin{framed}
    \noindent \textbf{Exercise 3.14: }
    
    \medskip
    
    
\end{framed}

\bigskip

\begin{framed}
    \noindent \textbf{Exercise 3.15: Lower bound for compare-and-swap based sorts}
    
    \medskip
    
    
\end{framed}

\bigskip

\begin{framed}
    \noindent \textbf{Exercise 3.16}
    
    \medskip
    
    
\end{framed}

\bigskip

\begin{framed}
    \noindent \textbf{Exercise 3.17}
    
    \medskip
    
    
\end{framed}

\bigskip

\begin{framed}
    \noindent \textbf{Exercise 3.18}
    
    \medskip
    
    
\end{framed}

\bigskip

\begin{framed}
    \noindent \textbf{Exercise 3.19}
    
    \medskip
    
    
\end{framed}

\bigskip

\begin{framed}
    \noindent \textbf{Exercise 3.20}
    
    \medskip
    
    
\end{framed}

\bigskip

\begin{framed}
    \noindent \textbf{Exercise 3.21}
    
    \medskip
    
    
\end{framed}

\bigskip

\begin{framed}
    \noindent \textbf{Exercise 3.22}
    
    \medskip
    
    
\end{framed}

\bigskip

\begin{framed}
    \noindent \textbf{Exercise 3.23}
    
    \medskip
    
    
\end{framed}

\bigskip

\begin{framed}
    \noindent \textbf{Exercise 3.24}
    
    \medskip
    
    
\end{framed}

\bigskip

\begin{framed}
    \noindent \textbf{Exercise 3.25}
    
    \medskip
    
    
\end{framed}

\bigskip

\begin{framed}
    \noindent \textbf{Exercise 3.26}
    
    \medskip
    
    
\end{framed}

\bigskip

\begin{framed}
    \noindent \textbf{Exercise 3.27}
    
    \medskip
    
    
\end{framed}

\bigskip

\begin{framed}
    \noindent \textbf{Exercise 3.28}
    
    \medskip
    
    
\end{framed}

\bigskip

\begin{framed}
    \noindent \textbf{Exercise 3.29}
    
    \medskip
    
    
\end{framed}

\bigskip

\begin{framed}
    \noindent \textbf{Exercise 3.30}
    
    \medskip
    
    
\end{framed}

\bigskip

\begin{framed}
    \noindent \textbf{Exercise 3.31}
    
    \medskip
    
    
\end{framed}

\bigskip

\begin{framed}
    \noindent \textbf{Exercise 3.32}
    
    \medskip
    
    
\end{framed}

\subsection*{Chapter 4: Quantum circuits}


\bigskip

\begin{framed}
    \noindent \textbf{Exercise 4.1}
    
    \medskip
    
    
\end{framed}


\bigskip

\begin{framed}
    \noindent \textbf{Exercise 4.2}
    
    \medskip
    
    
\end{framed}


\bigskip

\begin{framed}
    \noindent \textbf{Exercise 4.3}
    
    \medskip
    
    
\end{framed}


\bigskip

\begin{framed}
    \noindent \textbf{Exercise 4.4}
    
    \medskip
    
    
\end{framed}


\bigskip

\begin{framed}
    \noindent \textbf{Exercise 4.5}
    
    \medskip
    
    
\end{framed}


\bigskip

\begin{framed}
    \noindent \textbf{Exercise 4.6}
    
    \medskip
    
    
\end{framed}


\bigskip

\begin{framed}
    \noindent \textbf{Exercise 4.7}
    
    \medskip
    
    
\end{framed}


\bigskip

\begin{framed}
    \noindent \textbf{Exercise 4.8}
    
    \medskip
    
    
\end{framed}


\bigskip

\begin{framed}
    \noindent \textbf{Exercise 4.9}
    
    \medskip
    
    
\end{framed}


\bigskip

\begin{framed}
    \noindent \textbf{Exercise 4.10}
    
    \medskip
    
    
\end{framed}


\bigskip

\begin{framed}
    \noindent \textbf{Exercise 4.11}
    
    \medskip
    
    
\end{framed}


\bigskip

\begin{framed}
    \noindent \textbf{Exercise 4.12}
    
    \medskip
    
    
\end{framed}


\bigskip

\begin{framed}
    \noindent \textbf{Exercise 4.13}
    
    \medskip
    
    
\end{framed}


\bigskip

\begin{framed}
    \noindent \textbf{Exercise 4.14}
    
    \medskip
    
    
\end{framed}


\bigskip

\begin{framed}
    \noindent \textbf{Exercise 4.15}
    
    \medskip
    
    
\end{framed}


\bigskip

\begin{framed}
    \noindent \textbf{Exercise 4.16}
    
    \medskip
    
    
\end{framed}


\bigskip

\begin{framed}
    \noindent \textbf{Exercise 4.17}
    
    \medskip
    
    
\end{framed}


\bigskip

\begin{framed}
    \noindent \textbf{Exercise 4.18}
    
    \medskip
    
    
\end{framed}


\bigskip

\begin{framed}
    \noindent \textbf{Exercise 4.19}
    
    \medskip
    
    
\end{framed}


\bigskip

\begin{framed}
    \noindent \textbf{Exercise 4.20}
    
    \medskip
    
    
\end{framed}


\bigskip

\begin{framed}
    \noindent \textbf{Exercise 4.21}
    
    \medskip
    
    
\end{framed}


\bigskip

\begin{framed}
    \noindent \textbf{Exercise 4.22}
    
    \medskip
    
    
\end{framed}


\bigskip

\begin{framed}
    \noindent \textbf{Exercise 4.23}
    
    \medskip
    
    
\end{framed}


\bigskip

\begin{framed}
    \noindent \textbf{Exercise 4.24}
    
    \medskip
    
    
\end{framed}


\bigskip

\begin{framed}
    \noindent \textbf{Exercise 4.25}
    
    \medskip
    
    
\end{framed}


\bigskip

\begin{framed}
    \noindent \textbf{Exercise 4.26}
    
    \medskip
    
    
\end{framed}


\bigskip

\begin{framed}
    \noindent \textbf{Exercise 4.27}
    
    \medskip
    
    
\end{framed}


\bigskip

\begin{framed}
    \noindent \textbf{Exercise 4.28}
    
    \medskip
    
    
\end{framed}


\bigskip

\begin{framed}
    \noindent \textbf{Exercise 4.29}
    
    \medskip
    
    
\end{framed}


\bigskip

\begin{framed}
    \noindent \textbf{Exercise 4.30}
    
    \medskip
    
    
\end{framed}


\bigskip

\begin{framed}
    \noindent \textbf{Exercise 4.31}
    
    \medskip
    
    
\end{framed}


\bigskip

\begin{framed}
    \noindent \textbf{Exercise 4.32}
    
    \medskip
    
    
\end{framed}


\bigskip

\begin{framed}
    \noindent \textbf{Exercise 4.33}
    
    \medskip
    
    
\end{framed}


\bigskip

\begin{framed}
    \noindent \textbf{Exercise 4.34}
    
    \medskip
    
    
\end{framed}


\bigskip

\begin{framed}
    \noindent \textbf{Exercise 4.35}
    
    \medskip
    
    
\end{framed}


\bigskip

\begin{framed}
    \noindent \textbf{Exercise 4.36}
    
    \medskip
    
    
\end{framed}


\bigskip

\begin{framed}
    \noindent \textbf{Exercise 4.27}
    
    \medskip
    
    
\end{framed}


\bigskip

\begin{framed}
    \noindent \textbf{Exercise 4.38}
    
    \medskip
    
    
\end{framed}


\bigskip

\begin{framed}
    \noindent \textbf{Exercise 4.39}
    
    \medskip
    
    
\end{framed}


\bigskip

\begin{framed}
    \noindent \textbf{Exercise 4.40}
    
    \medskip
    
    
\end{framed}


\bigskip

\begin{framed}
    \noindent \textbf{Exercise 4.41}
    
    \medskip
    
    
\end{framed}


\bigskip

\begin{framed}
    \noindent \textbf{Exercise 4.42}
    
    \medskip
    
    
\end{framed}


\bigskip

\begin{framed}
    \noindent \textbf{Exercise 4.43}
    
    \medskip
    
    
\end{framed}


\bigskip

\begin{framed}
    \noindent \textbf{Exercise 4.44}
    
    \medskip
    
    
\end{framed}


\bigskip

\begin{framed}
    \noindent \textbf{Exercise 4.45}
    
    \medskip
    
    
\end{framed}


\bigskip

\begin{framed}
    \noindent \textbf{Exercise 4.46}
    
    \medskip
    
    
\end{framed}


\bigskip

\begin{framed}
    \noindent \textbf{Exercise 4.47}
    
    \medskip
    
    
\end{framed}


\bigskip

\begin{framed}
    \noindent \textbf{Exercise 4.48}
    
    \medskip
    
    
\end{framed}


\bigskip

\begin{framed}
    \noindent \textbf{Exercise 4.49}
    
    \medskip
    
    
\end{framed}


\bigskip

\begin{framed}
    \noindent \textbf{Exercise 4.50}
    
    \medskip
    
    
\end{framed}


\bigskip

\begin{framed}
    \noindent \textbf{Exercise 4.51}
    
    \medskip
    
    
\end{framed}

\subsection*{Chapter 5: The quantum Fourier transform and its applications}
\subsection*{Chapter 6: Quantum search algorithms}
\subsection*{Chapter 7: Quantum computers: physical realization}
\subsection*{Chapter 8: Quantum noise and quantum operations}
\subsection*{Chapter 9: Distance measures for quantum information }
\subsection*{Chapter 10: Quantum error-correction}
\subsection*{Chapter 11: Entropy and information}
\subsection*{Chapter 12: Quantum information theory }

\end{document}
