\documentclass{article}
\usepackage{amsmath, amssymb, framed}  
\usepackage{amsthm}            
\usepackage{graphicx}          
\usepackage{hyperref}          
\usepackage{enumitem}     
\usepackage{braket}
\usepackage[margin=1in]{geometry}  



\title{Chapter 2}
\author{Vasilii Nikonov}
\date{February 2025}

\begin{document}

\maketitle

\begin{framed} % Creates a visually separated block
    \noindent \textbf{Exercise 2.1: (Linear Dependence Example)}
    
    \medskip
    We can observe that:
    $$
    (1, -1) + (1, 2) - (2, 1) = (0,0) = \boldsymbol{0}
    $$
    Thus, the set of three provided vectors is linearly dependent.
\end{framed}

\bigskip

\begin{framed}
    \noindent \textbf{Exercise 2.2: (Matrix Representations: Example)}

    \medskip
    $V$ is a vector space with basis $\{ \ket{0}, \ket{1} \}$. \\
    $$
        A\ket{0} = \ket{1} \implies A\begin{bmatrix} 1 \\ 0 \end{bmatrix} = \begin{bmatrix} 0 \\ 1 \end{bmatrix} \implies \text{ first column of A is } 
        \ket{1}
    $$
    as multiplication of a 2 by 2 matrix by  $\ket{0}$ is just extracting the first column. Also
    $$
        A\ket{1} = \ket{0} \implies A\begin{bmatrix} 0 \\ 1 \end{bmatrix} = \begin{bmatrix} 1 \\ 0 \end{bmatrix} \implies \text{ second column of A is } \ket{0}
    $$
    Thus $A = \begin{bmatrix} 0 & 1 \\ 1 & 0 \end{bmatrix}$. \\ \\ 
    We know, from (2.7), that $\ket{v_1} = \ket{+} = \begin{bmatrix} \frac{1}{\sqrt{2}} \\ \frac{1}{\sqrt{2}} \end{bmatrix}$ and $\ket{v_2} = \ket{-} = \begin{bmatrix} \frac{1}{\sqrt{2}} \\ \frac{-1}{\sqrt{2}} \end{bmatrix}$ span the $\mathbb{C}^{2}$, they also form a basis for $\mathbb{C}^2$, as they are linearly independent (the only solution for $c_1\ket{+} + c_2\ket{-} = 0$ is trivial). \\ \\ 
    So we can form following linear operators $A_i$ from $\mathbb{C}^{2}$ to $\mathbb{C}^{2}$ :
    \begin{enumerate}
        \item $A_1\ket{0} = \ket{+}$ and $A_1\ket{1} = \ket{-}$
        \item $A_2\ket{0} = \ket{-}$ and $A_2\ket{1} = \ket{+}$
        \item $A_3\ket{+} = \ket{0}$ and $A_3\ket{-} = \ket{1}$
        \item $A_4\ket{+} = \ket{1}$ and $A_4\ket{-} = \ket{0}$
    \end{enumerate}
    We can compute $A_2 = \begin{bmatrix} a & b \\ c & d \end{bmatrix}$, where $ \begin{bmatrix} a & b \\ c & d \end{bmatrix} \begin{bmatrix} 1 \\ 0 \end{bmatrix} = \begin{bmatrix} \frac{1}{\sqrt{2}} \\ \frac{-1}{\sqrt{2}} \end{bmatrix}$ and $ \begin{bmatrix} a & b \\ c & d \end{bmatrix} \begin{bmatrix} 0 \\ 1 \end{bmatrix} = \begin{bmatrix} \frac{1}{\sqrt{2}} \\ \frac{1}{\sqrt{2}} \end{bmatrix}$. We can trivially see, that $A_2 = \begin{bmatrix} \frac{1}{\sqrt{2}} & \frac{1}{\sqrt{2}} \\ \frac{-1}{\sqrt{2}} & \frac{1}{\sqrt{2}} \end{bmatrix}$ \\ \\ 
    A case $A_3 = \begin{bmatrix} e & f \\ g & h \end{bmatrix}$ is a bit more interesting, because we cannot just extract the columns. We can write in matrix form.
    $$
    \begin{bmatrix} e & f \\ g & h \end{bmatrix} \begin{bmatrix} \frac{1}{\sqrt{2}} && \frac{1}{\sqrt{2}} \\ \frac{1}{\sqrt{2}} && \frac{-1}{\sqrt{2}} \end{bmatrix} = I
    $$
    We can multiply both sides on the right by a transpose of the second matrix, as it is orthogonal.
    $$
    A_3 = \begin{bmatrix} e & f \\ g & h \end{bmatrix} = \begin{bmatrix} \frac{1}{\sqrt{2}} && \frac{1}{\sqrt{2}} \\ \frac{1}{\sqrt{2}} && \frac{-1}{\sqrt{2}} \end{bmatrix}
    $$
\end{framed}

\bigskip

\begin{framed}
    \noindent \textbf{Exercise 2.3: (Matrix representation for operator products)}

    \medskip
    We have the following combination of linear operators and vector spaces $V \xrightarrow{A} W \xrightarrow{B} X$. From (2.12) we can write: 
    \begin{equation}
        A\ket{v_i} = \sum_{j}{A_{ji}\ket{w_j}}\text{, and } B\ket{w_j} = \sum_{k}{B_{kj}\ket{x_k}} 
        \tag{1}
    \end{equation}
    $$BA\ket{v_j} = B(\sum_{j}{A_{ji}\ket{v_j}}) \text{ from (1)}$$
    $$
    B(\sum_{j}{A_{ji}\ket{v_j}}) = \sum_{j}{A_{ji}B(\ket{v_j}}) \text{ from linearity of inputs in (2.10)}
    $$
    $$
     = \sum_{j}{A_{ji}\sum_{k}{B_{kj}\ket{x_k}}} \text{ from (1)}
    $$
    \begin{equation}
    = \sum_k(\sum_j{B_{kj}A_{ji}) \ket{x_k}} \text{ rearranging the sum order}
    \tag{2}
    \end{equation}
    If we look from the point of view of a linear operation from V to X, then we need to have some matrix $C$, that would give us:
    $$C\ket{v_i} = \sum_k{C_{ki}\ket{x_k}}$$
    It is precisely 
    $$
    BA_{ki} = \sum_j{B_{kj}A_{ji}} \text{ from (2)}
    $$
\end{framed}

\bigskip

\begin{framed}
    \noindent \textbf{Exercise 2.4: (Matrix representation for identity)}

    \medskip
    
\end{framed}

\bigskip

\begin{framed}
    \noindent \textbf{Exercise 2.5}

    \medskip
\end{framed}

\bigskip

\begin{framed}
    \noindent \textbf{Exercise 2.6}
    
    \medskip
    
    Here we need to show, that any inner product $(\cdot, \cdot)$ is conjugate-linear in the first argument.
    $$
    (\sum_{i}{\lambda_i \ket{w_i}}, \ket{v}) = (\ket{v}, \sum_{i}{\lambda_i \ket{w_i}})^{*} \text{ from (2.13 (2))}
    $$
    $$
    = (\sum_{i}{\lambda_i(\ket{v}, \ket{w_i})})^{*} \text{ from linearity of second inner-product argument (2.13 (1))}
    $$
    $$
    = \sum_{i}{(\lambda_i(\ket{v}, \ket{w_i}))^*} \text{ as conjugate of a sum is a sum of conjugates}
    $$
    $$
    = \sum_{i}{\lambda_i^*(\ket{v}, \ket{w_i})^*}  \text{ as conjugate of a product is a product of conjugates}
    $$
    $$
    = \sum_{i}{\lambda_i^*(\ket{w_i}, \ket{v})} \text{ from (2.13 (2))}
    $$
    
\end{framed}

\bigskip

\begin{framed}
    \noindent \textbf{Exercise 2.7}
    
    \medskip
    
    
\end{framed}

\bigskip

\begin{framed}
    \noindent \textbf{Exercise 2.8}
    
    \medskip
    
    
\end{framed}

\bigskip

\begin{framed}
    \noindent \textbf{Exercise 2.9: (Pauli operators and the outer product)}
    
    \medskip
    
    
\end{framed}

\bigskip

\begin{framed}
    \noindent \textbf{Exercise 2.10}
    
    \medskip
    
    
\end{framed}

\bigskip

\begin{framed}
    \noindent \textbf{Exercise 2.11: (Eigendecomposition of the Pauli matrices)}
    
    \medskip
    
    
\end{framed}

\bigskip

\begin{framed}
    \noindent \textbf{Exercise 2.12}
    
    \medskip
    
    
\end{framed}

\bigskip

\begin{framed}
    \noindent \textbf{Exercise 2.13}
    
    \medskip
    
    
\end{framed}

\bigskip

\begin{framed}
    \noindent \textbf{Exercise 2.14: (Anti-linearity of the adjoint)}
    
    \medskip
    
    
\end{framed}

\bigskip

\begin{framed}
    \noindent \textbf{Exercise 2.15}
    
    \medskip
    
    
\end{framed}

\bigskip

\begin{framed}
    \noindent \textbf{Exercise 2.16}
    
    \medskip
    
    
\end{framed}

\bigskip

\begin{framed}
    \noindent \textbf{Exercise 2.17}
    
    \medskip
    
    
\end{framed}

\bigskip

\begin{framed}
    \noindent \textbf{Exercise 2.18}
    
    \medskip
    
    
\end{framed}

\bigskip

\begin{framed}
    \noindent \textbf{Exercise 2.19: (Pauli matrices: Hermitian and unitary)}
    
    \medskip
    Matrix A is Hermitian if $A^{\dagger} = A$, matrix B is unitary if $B^{\dagger}B = BB^{\dagger} = I$ \\\\
    Now let's consider each of the Pauli matrices 
    \begin{enumerate}
        \item $\sigma_0 = I = \begin{bmatrix} 1 & 0 \\ 0 & 1 \end{bmatrix}$\\\\
        $I^{\dagger} = (I^{T})^{*} = I $ as identity is symmetric, and complex conjugate of a real value is just itself, thus I is Hermitian. \\\\
        $I^{\dagger}I = II = I = II^{\dagger} \implies I$ is unitary

        \item $\sigma_1=\sigma_x=X = \begin{bmatrix} 0 & 1 \\ 1 & 0 \end{bmatrix}$\\\\
        $X^{\dagger} = (X^{T})^* = \begin{bmatrix} 0 & 1 \\ 1 & 0 \end{bmatrix}^{*} = X \implies X$ is Hermitian\\\\
        $X^{\dagger}X = XX = XX^{\dagger} = \begin{bmatrix} 0 & 1 \\ 1 & 0 \end{bmatrix} \begin{bmatrix} 0 & 1 \\ 1 & 0 \end{bmatrix} = \begin{bmatrix} 1 & 0 \\ 0 & 1 \end{bmatrix} = I \implies X$ is unitary

        

        \item $\sigma_2=\sigma_y=Y = \begin{bmatrix} 0 & -i \\ i & 0 \end{bmatrix}$\\\\
        $Y^{\dagger} = (Y^{T})^* = \begin{bmatrix} 0 & i \\ -i & 0 \end{bmatrix}^{*} = \begin{bmatrix} 0 & -i \\ i & 0 \end{bmatrix} = Y \implies Y$ is Hermitian\\\\
        $Y^{\dagger}Y = YY = YY^{\dagger} = \begin{bmatrix} 0 & -i \\ i & 0 \end{bmatrix}\begin{bmatrix} 0 & -i \\ i & 0 \end{bmatrix}=\begin{bmatrix} -i^2 & 0 \\ 0 & -i^2 \end{bmatrix}=\begin{bmatrix} 1 & 0 \\ 0 & 1 \end{bmatrix} = I \implies Y$ is unitary

        \item $\sigma_3=\sigma_z=Z = \begin{bmatrix} 1 & 0 \\ 0 & -1 \end{bmatrix}$\\\\
        $Z^{\dagger} = (Z^{T})^{*} = \begin{bmatrix} 1 & 0 \\ 0 & -1 \end{bmatrix}^{*} = Z \implies Z$ is Hermitian\\\\
        $Z^{\dagger}Z = ZZ = ZZ^{\dagger} = \begin{bmatrix} 1 & 0 \\ 0 & -1 \end{bmatrix}\begin{bmatrix} 1 & 0 \\ 0 & -1 \end{bmatrix}=\begin{bmatrix} 1 & 0 \\ 0 & (-1)^2 \end{bmatrix}=I \implies Z$ is unitary
    \end{enumerate}
    
\end{framed}

\bigskip

\begin{framed}
    \noindent \textbf{Exercise 2.20: (Basis changes)}
    
    \medskip
    
    
\end{framed}

\bigskip

\begin{framed}
    \noindent \textbf{Exercise 2.21}
    
    \medskip
    
    
\end{framed}

\bigskip

\begin{framed}
    \noindent \textbf{Exercise 2.22}
    
    \medskip
    
    
\end{framed}

\bigskip

\begin{framed}
    \noindent \textbf{Exercise 2.23}
    
    \medskip
    
    
\end{framed}

\bigskip

\begin{framed}
    \noindent \textbf{Exercise 2.24: (Hermiticity of positive operators)}
    
    \medskip
    
    
\end{framed}

\bigskip

\begin{framed}
    \noindent \textbf{Exercise 2.25}
    
    \medskip
    
    
\end{framed}

\bigskip

\begin{framed}
    \noindent \textbf{Exercise 2.26}
    
    \medskip
    
    
\end{framed}

\bigskip

\begin{framed}
    \noindent \textbf{Exercise 2.27}
    
    \medskip
    
    
\end{framed}

\bigskip

\begin{framed}
    \noindent \textbf{Exercise 2.28}
    
    \medskip
    
    
\end{framed}

\bigskip

\begin{framed}
    \noindent \textbf{Exercise 2.29}
    
    \medskip
    
    
\end{framed}

\bigskip

\begin{framed}
    \noindent \textbf{Exercise 2.30}
    
    \medskip
    
    
\end{framed}

\bigskip

\begin{framed}
    \noindent \textbf{Exercise 2.31}
    
    \medskip
    
    
\end{framed}

\bigskip

\begin{framed}
    \noindent \textbf{Exercise 2.32}
    
    \medskip
    
    
\end{framed}

\bigskip

\begin{framed}
    \noindent \textbf{Exercise 2.33}
    
    \medskip
    
    
\end{framed}

\bigskip

\begin{framed}
    \noindent \textbf{Exercise 2.34}
    
    \medskip
    
    
\end{framed}

\bigskip

\begin{framed}
    \noindent \textbf{Exercise 2.35: (Exponential of the Pauli matrices)}
    
    \medskip
    
    
\end{framed}

\bigskip

\begin{framed}
    \noindent \textbf{Exercise 2.36}
    
    \medskip
    
    
\end{framed}

\bigskip

\begin{framed}
    \noindent \textbf{Exercise 2.37: (Cyclic property of the trace)}
    
    \medskip
    
    
\end{framed}

\bigskip

\begin{framed}
    \noindent \textbf{Exercise 2.38: (Linearity of the trace)}
    
    \medskip
    
    
\end{framed}

\bigskip

\begin{framed}
    \noindent \textbf{Exercise 2.39: (The Hilbert–Schmidt inner product on operators)}
    
    \medskip
    
    
\end{framed}

\bigskip

\begin{framed}
    \noindent \textbf{Exercise 2.40: (Commutation relations for the Pauli matrices)}
    
    \medskip
    
    
\end{framed}

\bigskip

\begin{framed}
    \noindent \textbf{Exercise 2.41: (Anti-commutation relations for the Pauli matrices)}
    
    \medskip
    
    
\end{framed}

\bigskip

\begin{framed}
    \noindent \textbf{Exercise 2.42}
    
    \medskip
    $$
    \text{Consider }  \frac{[A,B] +\{A,B\}}{2} = \frac{AB - BA + AB + BA}{2} \text{ from (2.66), (2.67)}
    $$
    $$
    = \frac{2AB}{2} = AB
    $$
    
\end{framed}

\bigskip

\begin{framed}
    \noindent \textbf{Exercise 2.43}
    
    \medskip
    
    
\end{framed}

\bigskip

\begin{framed}
    \noindent \textbf{Exercise 2.44}
    $$
    [A,B] = 0 \implies AB - BA = 0, \{A, B\} = 0 \implies AB + BA = 0 \text{ from definitions (2.66, 2.67)}
    $$
    $$
    \text{Then their sum } [A,B] + \{A,B\} = 2AB = 0 \text{ We now multiply both sides on the left by } A^{-1} \text{}
    $$
    $$
    A^{-1}2AB = A^{-1}0 \implies 2B = 0 \implies B = 0.
    $$
    \medskip
    
    
\end{framed}


\bigskip

\begin{framed}
    \noindent \textbf{Exercise 2.45}
    
    \medskip
    
    Consider Hermitian conjugate of a commutator between two operators A and B 
    $$
    [A,B]^{\dagger} = (AB - BA)^{\dagger} \text{ from (2.66)}
    $$
    $$
    = ((AB - BA)^{T})^{*} \text{ from definition of Hermitian conjugate}
    $$
    $$
    = ((AB)^{T} - (BA)^{T}) ^ {*} = (B^TA^T - A^TB^T)^* = (B^T)^*(A^T)^* - (A^T)^*(B^T)^*
    $$
    $$
    = B^{\dagger}A^{\dagger} - A^{\dagger}B^{\dagger} = [B^{\dagger},A^{\dagger}]
    $$
\end{framed}

\bigskip
\begin{framed}
\noindent \textbf{Exercise 2.46}

\medskip

Consider $[A, B] = AB - BA = -(-AB + BA) = - (BA - AB) = -[B, A]$
\end{framed}

\bigskip

\begin{framed}
    \noindent \textbf{Exercise 2.47}
    
    \medskip

    $$
    A, B \text{ are Hermitian, then each is equal to their conjugate transpose } A = A^{\dagger}, B = B^{\dagger}
    $$
    $$
    \text{Consider } (i[A, B])^{\dagger} = i^{*}[A,B]^{\dagger} \text{ as } (cA)^{\dagger} = c^{*}A^{\dagger} \text{ where } c \text{ is a complex scalar}.
    $$
    $$
    = (-i)[A, B]^{\dagger} = (-i)[B^{\dagger},A^{\dagger}] \text{ from Exercise 2.45}
    $$
    $$
    = (-i)[B, A] = -(i)(-[A,B]) \text{ from exercise 2.46} = i[A,B] \implies i[A,B] \text{ is Hermitian}
    $$
\end{framed}

\bigskip

\begin{framed}
    \noindent \textbf{Exercise 2.48}
    
    \medskip
    
    
\end{framed}

\bigskip

\begin{framed}
    \noindent \textbf{Exercise 2.49}
    
    \medskip
    
    
\end{framed}

\bigskip

\begin{framed}
    \noindent \textbf{Exercise 2.50}
    
    \medskip
    
    
\end{framed}

\bigskip

\begin{framed}
    \noindent \textbf{Exercise 2.51}
    
    \medskip
    
    
\end{framed}

\bigskip

\begin{framed}
    \noindent \textbf{Exercise 2.52}
    
    \medskip
    
    
\end{framed}

\bigskip

\begin{framed}
    \noindent \textbf{Exercise 2.53}
    
    \medskip
    
    
\end{framed}

\bigskip

\begin{framed}
    \noindent \textbf{Exercise 2.54}
    
    \medskip
    
    
\end{framed}

\bigskip

\begin{framed}
    \noindent \textbf{Exercise 2.55}
    
    \medskip
    
    
\end{framed}

\bigskip

\begin{framed}
    \noindent \textbf{Exercise 2.56}
    
    \medskip
    
    
\end{framed}

\bigskip

\begin{framed}
    \noindent \textbf{Exercise 2.57: (Cascaded measurements are single measurements)}
    
    \medskip
    
    
\end{framed}

\bigskip

\begin{framed}
    \noindent \textbf{Exercise 2.58}
    
    \medskip
    
    
\end{framed}

\bigskip

\begin{framed}
    \noindent \textbf{Exercise 2.59}
    
    \medskip
    
    
\end{framed}

\bigskip

\begin{framed}
    \noindent \textbf{Exercise 2.60}
    
    \medskip
    
    
\end{framed}

\bigskip

\begin{framed}
    \noindent \textbf{Exercise 2.61}
    
    \medskip
    
    
\end{framed}

\bigskip

\begin{framed}
    \noindent \textbf{Exercise 2.62}
    
    \medskip
    
    
\end{framed}

\bigskip

\begin{framed}
    \noindent \textbf{Exercise 2.63}
    
    \medskip
    
    
\end{framed}

\bigskip

\begin{framed}
    \noindent \textbf{Exercise 2.64}
    
    \medskip
    
    
\end{framed}

\bigskip

\begin{framed}
    \noindent \textbf{Exercise 2.65}
    
    \medskip
    
    
\end{framed}

\bigskip

\begin{framed}
    \noindent \textbf{Exercise 2.66}
    
    \medskip
    
    
\end{framed}

\bigskip

\begin{framed}
    \noindent \textbf{Exercise 2.67}
    
    \medskip
    
    
\end{framed}

\bigskip

\begin{framed}
    \noindent \textbf{Exercise 2.68}
    
    \medskip
    
    
\end{framed}

\bigskip

\begin{framed}
    \noindent \textbf{Exercise 2.69}
    
    \medskip
    
    
\end{framed}

\bigskip

\begin{framed}
    \noindent \textbf{Exercise 2.70}
    
    \medskip
    
    
\end{framed}

\bigskip

\begin{framed}
    \noindent \textbf{Exercise 2.71: (Criterion to decide if a state is mixed or pure)}
    
    \medskip
    
    
\end{framed}

\bigskip

\begin{framed}
    \noindent \textbf{Exercise 2.72: (Bloch sphere for mixed states)}
    
    \medskip
    
    
\end{framed}

\bigskip

\begin{framed}
    \noindent \textbf{Exercise 2.73}
    
    \medskip
    
    
\end{framed}

\bigskip

\begin{framed}
    \noindent \textbf{Exercise 2.74}
    
    \medskip
    
    
\end{framed}

\bigskip

\begin{framed}
    \noindent \textbf{Exercise 2.75}
    
    \medskip
    
    
\end{framed}

\bigskip

\begin{framed}
    \noindent \textbf{Exercise 2.76}
    
    \medskip
    
    
\end{framed}

\bigskip

\begin{framed}
    \noindent \textbf{Exercise 2.77}
    
    \medskip
    
    
\end{framed}

\bigskip

\begin{framed}
    \noindent \textbf{Exercise 2.78}
    
    \medskip
    
    
\end{framed}

\bigskip

\begin{framed}
    \noindent \textbf{Exercise 2.79}
    
    \medskip
    
    
\end{framed}

\bigskip

\begin{framed}
    \noindent \textbf{Exercise 2.80}
    
    \medskip
    
    
\end{framed}

\bigskip

\begin{framed}
    \noindent \textbf{Exercise 2.81: (Freedom in purifications)}
    
    \medskip
    
    
\end{framed}

\bigskip

\begin{framed}
    \noindent \textbf{Exercise 2.82}
    
    \medskip
    
    
\end{framed}

\bigskip

\begin{framed}
    \noindent \textbf{End of Chapter Exercise 2.1: (Functions of the Pauli matrices)}
    
    \medskip
    
    
\end{framed}

\bigskip

\begin{framed}
    \noindent \textbf{End of Chapter Exercise 2.2: (Properties of the Schmidt number)}
    
    \medskip
    
    
\end{framed}

\bigskip

\begin{framed}
    \noindent \textbf{End of Chapter Exercise 2.3: (Tsirelson’s inequality)}
    
    \medskip
    
    
\end{framed}

\end{document}
